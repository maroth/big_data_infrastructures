\documentclass[a4paper]{article}

\usepackage{listings}
\usepackage[T1]{fontenc}
\usepackage{color}
\usepackage{pxfonts}

\definecolor{dkgreen}{rgb}{0,0.6,0}
\definecolor{gray}{rgb}{0.5,0.5,0.5}
\definecolor{mauve}{rgb}{0.58,0,0.82}

\lstset{frame=tb,
    aboveskip=3mm,
    belowskip=3mm,
    showstringspaces=false,
    columns=flexible,
    basicstyle={\small\ttfamily},
    numbers=none,
    numberstyle=\tiny\color{gray},
    keywordstyle=\color{blue},
    commentstyle=\color{dkgreen},
    stringstyle=\color{mauve},
    breaklines=true,
    breakatwhitespace=true,
    tabsize=3
}

\title{Big Data Infrastructures - Homework 2: SQL Review Part 2}

\author{Markus Roth}

\begin{document}
\maketitle

\section{A: Indexes}
\subsection{Create Table}

\lstinputlisting{position.sql}

\section{Populate Table}

\lstinputlisting{populate.py}

\subsection{Choose Rectangular Area}

\lstinputlisting{ii.sql}

\subsection{Describe Execution Plan}

\lstinputlisting{explain_output.txt}

Seq scan on "position p" means that the database engine must start reading at the beginning of the table and check each and every row of the table. If the row fullfills the conditions, it will be part of the result set, otherwise it won't be. 

The execution time of the query is around 20 ms.

\subsection{Create Indexes}

\lstinputlisting{create_indexes.sql}

\subsection{Re-Check Execution Plan}

\lstinputlisting{explain_output_with_indexes.txt}

The new execution first uses a Bitmap Heap Scan on "position p", where it filters for the lat condition. After that, it uses a Bitmap Index Scan over the lon\textunderscore index on the lon condition.

The execution time of the query is now around 7 ms, quite an improvement!

\subsection{Bitmap Index Scan}

A Bitmap Index Scan optimizes an index scan to only load disk pages once each. This is good if there is some data locality, but not necessarity good if the data is completely randomly stored \footnote{url{http://stackoverflow.com/questions/6592626/what-is-a-bitmap-heap-scan-in-a-query-plan}}.

\section{B: Relational Queries}

\subsection{i}

\lstinputlisting{b_i.sql}

\subsection{ii}

\lstinputlisting{b_ii.sql}

\subsection{iii}

\lstinputlisting{b_iii.sql}

\subsection{iv}

\lstinputlisting{b_iv.sql}

\subsection{v}

This one is really frying my brain :(

This is as far as I got, but it's not valid syntax.

\lstinputlisting{b_v.sql}
\end{document}
